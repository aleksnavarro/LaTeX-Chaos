\documentclass[12pt]{article}
\usepackage[spanish,es-tabla]{babel}
\usepackage[utf8]{inputenc}
\usepackage{amsmath}
\usepackage{graphicx}
\usepackage{fullpage}
\usepackage{amssymb}
\usepackage{amsthm}
\usepackage{fancyvrb}
\usepackage{lipsum,multicol}
\usepackage{framed}
\usepackage{enumerate}
\usepackage[shortlabels]{enumitem}
\usepackage[colorinlistoftodos]{todonotes}

%Modelo criado por Wellington Sales



\begin{document}

% portada
\begin{titlepage}
    \begin{center}
       
        {\Huge Titulo do Trabalho} \\[4.9cm]
        {\large Seu Nome} \\[3.5cm]
        \vfill
    \end{center}
\end{titlepage}

% índice
\tableofcontents
\cleardoublepage

%cuerpo del documento
\section*{Introducción}

\newpage
\section{Trigonometría}

\subsection{Angulos, arcos y sistemas de medición}

\subsection{Definición de las 6 funciones trigonométricas}

\subsection{Identidades fundamentales}

\subsection{Gráficas de funciones trigonométricas}

\subsection{Ley de senos}

\subsection{Ley de cosenos}

\subsection{Solución de triángulos}

\newpage
\section{Geometría análitica}

\subsection{La línea recta}

\subsection{La circunferencía}

\subsection{La parábola}

\subsection{La elipse}

\subsection{La hipérbola}




\end{document}
              