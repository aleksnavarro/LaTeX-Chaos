%%%%%%%%%%%%%%%%%%%%%%%%%%%%%%%%%%%%%%%%%%%%%%%%%%%%%%%%%%%%%%%%%%%%%%%%%%%%%%%%%
% TEXmaster: Master Document LaTex, Versión 0.3									%
%		Documento tipo article en doble columna.				  			   /%
% 	Para Laboratorio de óptica, FS11 D09.									///	%
% Diseñado en TexMaker 4.0.3, por A. Navarro, L. Santiago y R. Sandoval	  ///	%
%		Martes 8 de abril de 2014										///		%
%%%%%%%%%%%%%%%%%%%%%%%%%%%%%%%%%%%%%%%%%%%%%%%%%%%%%%%%%%%%%%%%%%%%%%%%%%%%%%%%%

\documentclass[12pt,letterpaper,openany,twocolumn,]{article} %Tipo de documento.
\usepackage[utf8]{inputenc} %Codificación.
\usepackage[spanish,es-tabla]{babel} %Idioma.
\usepackage{amsmath,times} %Paquete de escritura matematica.
\usepackage{amsfonts} %Paquete para fuentes de escritura matematica.
\usepackage{amssymb} %Paquete para fuente de símbolos AMS.
%\usepackage{aps}
\usepackage{multicol}
\usepackage{lipsum}
\usepackage{stfloats}
\usepackage{flushend}
\usepackage{graphicx} %Paquete para soporte gráfico.
\usepackage{caption}
\usepackage{verbatim} %Paquete para comentarios y citas.
\usepackage{bm} %Paquete para escribir símbolos matematicos en negritas.
%\usepackage[showframe]{geometry} %Especificaciones de margenes
\usepackage{appendix} %Paquete de apéndices.
%\usepackage[]{mcode} %Paquete para soporte de código.
\usepackage{color}
\sloppy
\definecolor{lightgray}{gray}{0.5}
\setlength{\parindent}{0pt}
\parskip 2ex
\textwidth =6.5in
\textheight =8in
\oddsidemargin =0.0in
\evensidemargin =0.0in
\usepackage{sectsty} %Paquete para personalización de títulos de secciones y subsecciones.
%\numberwithin{equation}{chapter} %Númeración de ecuaciones
\sectionfont{\large} %Tamaño de fuente de sección, %\small o %\large.
\author{autor} %autor
\title{título} %título
\date{lugar\\afiliación\\fecha} %fecha del documento, %\today puede ser cambiado por una fecha manual.
%\date{lugar/institucion\\fecha} %Afiliación del autor y fecha del documento


\begin{document}

\maketitle %Hace el título del documento en base con la información escrita en %\author, %\title, %\affiliation y %\date.
\begin{center}\rule{0.9\textwidth}{0.1mm} \end{center}
\begin{abstract}

{\bf Palabras Clave:} %añade las palabras clave (después del "}" ).
\end{abstract}
\begin{center}\rule{0.9\textwidth}{0.1mm} \end{center}

%\tableofcontents %Crea una tabla de contenidos.
%\begin{multicols}{2}

\section{Introducción}

\section{Observaciones}

\section{Resultados}

\section{Discusión}

\section{Conclusiones}

\section{Bibliografia}

%\end{multicols}

\newpage
\appendix

\end{document}

%\begin{figure*}[htbp]
%\centering
%\includegraphics[width=1\textwidth]{./nombrearchivo.jpg}
%\caption{Piedefigura.}
%\label{fig:etiqueta}
%\end{figure*}

%\begin{center}
%\includegraphics[height=60mm]{./nombredelarchivo.jpg}
%\captionof{figure}{\footnotesize piedefoto.}
%\end{center}

%\begin{table}[h]
%\begin{center}
%\caption{nombre}
%\begin{tabular}[H]{|l|l|l|}
%\hline
%1 & 2 & 3 
%\\
%\hline
%4 & 4 & 6				
%\\
%\hline
%7 & 8 & 9
%\\
%\hline
%\end{tabular}
%\end{center}
%\end{table}